\documentclass[11pt]{article}

\usepackage[quiet]{fontspec}
\usepackage{xunicode}
\usepackage{xltxtra}
\usepackage{palatino}
\usepackage{tabularx}
\usepackage{longtable}
\usepackage{booktabs}
\usepackage{multirow}
\usepackage{fancyhdr}
\usepackage{xltxtra}
\usepackage{enumitem}
\usepackage[margin={.6in}]{geometry}
\linespread{1.05}

\defaultfontfeatures{Mapping=tex-text}

\pagestyle{fancy}
\fancyhead{}
\fancyfoot[C]{\small Resume Generated by \XeLaTeX}
\renewcommand{\headrulewidth}{0pt}

\newcommand{\georgia}{\fontspec{Georgia}}
\newcommand{\pala}{\fontspec{Palatino Linotype}}
\newcommand{\helv}{\fontspec{Helvetica}}

\setmainfont{Calibri}

\begin{document}

\thispagestyle{empty}

\begin{center}
\begin{tabular}{lr}
    \multirow{3}{*}{\makebox[.05\textwidth][l]{}\makebox[.55\textwidth][l]{\Huge \pala Hu Ziming}} & %
        \makebox[.35\textwidth][r]{\pala Tel: (+86)1861-832-8360} \\
      & \makebox[.35\textwidth][r]{\pala Mail: hzmangel@gmail.com} \\
      & \makebox[.35\textwidth][r]{\pala Blog: http://blog.hzmangel.info} \\
  \bottomrule
\end{tabular}
\end{center}

\centering

\begin{tabularx}{\textwidth}{lX}
    \makebox[.12\textwidth][l]{\large \bf Profile} &%
        Have near 4 years experience on developing IBM HPC scheduler software LoadLeveler,
        prime responsibility is implementing configuration and network related area for LoadLeveler. %
        Familiar with programming under Linux, have 9 years experience on using and administrating Linux.
        Familiar with object-oriented programming, good skills in C++ and Python. %
        Familiar with using Python scripts and Unix tools for administration tasks. %
        Have experienced on Python web frameworks Django/Tornado/bottle,
        have knowledge on html5/javascript programming.
        Have experienced on Haskell and Scala, have knowledge on functional programming model. \\\\
\end{tabularx}

\begin{tabularx}{\textwidth}{lX}
    \makebox[.12\textwidth][l]{\large \bf Patent} &%
    Published patent \textsc{\textit{\textbf{Scheduling Jobs In a Cluster}}} while working at IBM,
    publication number is {\bf US 2010/0223618 A1} \\\\
\end{tabularx}

\begin{tabularx}{\textwidth}{lX}
    \makebox[.12\textwidth][l]{\large \bf Working} & {\bf IBM staff software engineer} (Apr, 2010 -- current) \\
    \makebox[.12\textwidth][l]{\large \bf Experience} & {\bf IBM software engineer} (Apr, 2008 -- Apr, 2010) \\
\end{tabularx}

\setlength{\leftskip}{.155\textwidth}
Worked on several components of LoadLeveler, which is the scheduler in IBM high
performance computing software stack.  My prime areas are software configuration
and network communication. I have added shared memory, database and IPv6 support,
and designed a new mechanism for trouble shooting,
which can worked with current tool of Blue Gene system.
I have also help customers to solve the problem with L2 team.

\begin{description}[leftmargin=\leftskip, labelindent=.12\textwidth]
    \item[{Dynamic Resource Usage Modification \normalfont(Jan, 2012 - current):}] %
        Design and implement new feature for modifying resource usage of running job,
        which improves current code that can only modify idle job.
        I took charge of implementing update at job scheduler daemon,
        and reschedule the job in waiting list. This project is under development.

    \item[{Task Migration \normalfont(Jun, 2011 - Jan, 2012):}] %
        Design and implement the checkpoint/restart feature on Linux platform with lxc module,
        and add task migration feature to the code. This new feature removed the dependency of third-part library,
        and improve the stability of product. I have joined the project in the middle and late period,
        implemented data transfer between departure node arrival node,
        and help to complete the function in job schedule daemons.

    \item[{Log Enhancement \normalfont(Apr, 2010 - Apr, 2011):}] %
        Designed and implemented an enhanced log mechanism for examining the running status of LoadLeveler.
        The new mechanism adds well designed messages to record the running status besides current log system.
        New added log messages will be gathered into file or database for further analysis.
        This improvement provides a more quicker and effective method for tracing problem,
        which also improves the usability of the software.

    \item[{IPv6 Support \normalfont(Oct, 2009 - Mar, 2010):}] %
        Designed and implemented IPv6 support to the LoadLeveler by refractoring the infrastructure library.
        With this change, LoadLeveler can be deployed in pure IPv6 and IPv4/IPv6 dual stack environment,
        which improves the usable range of software and
        remove the limitation of cluster scale brought by amount of IPv4 address.

    \item[{Configuration Enhancement \normalfont(Dec, 2008 - Sep, 2009):}] %
        Designed and implemented a new method for saving configuration.
        The new method add a middle layer between LoadLeveler and configuration source,
        which encapsulates the underlying operations and provides common interfaces to the upper layer code.
        With this improvement, LoadLeveler can get configuration from variable sources currently.
        Another change in this method is saving parsed configuration content into shared memory instead of internal object.
        This change reduces the complexity of deploying LoadLeveler into large scale cluster.
        In addition, a new configuration dispatching mechanism has been introduced to the software for decentralizing the configuration server,
        which can reduce load of the server and improve the cluster's fault tolerance to SPOF.\\
\end{description}
\setlength{\leftskip}{0pt}


\begin{tabularx}{\textwidth}{lX}
    \parbox[t]{.12\textwidth}{\large \bf Community ~~~~~~~~~Activities} & % use ~ to disable word break
        \vspace{-1.7 em} % remove the space before the list
        \begin{description}
            \item[\textit{linuxfb} Community]
                Co-founder of BUPT \textit{linuxfb} group,
                organize monthly discussion and build website for publishing announcement and recording history by Django (http://linuxfb.net),
                also responsible to gather discussion topics and slides in the mail group.
            \item[Experience Sharing] Interest in new technologies,
                often share experience in the \textit{linuxfb} monthly discussion or blog.
                Topics contained Python, Google App Engine, Hadoop, Django and Haskell.
            \item[{System Administrator}] %
                Responsible to configuring and maintaining developing environment and virtualization platform in the team from Apr, 2009.
            \item[Translation] Completed the translation of {\it The Django Book} with community members.
            \item[OCFS2] Submitted two Python related patches to OCFS2 project. (BNC\#476388 and BNC\#448523)
            \item[YAST2] Submitted a Python proxy module for yast2-multipath package (http://goo.gl/xFPiJ)
        \end{description}\\\\
\end{tabularx}

\begin{tabularx}{\textwidth}{lX}
    \makebox[.12\textwidth][l]{\large \bf Education} & {\bf Beijing University of Posts and Telecommunications, Master of engineering} \\
        & School of Information Engineering, Signal and Information Processing, From Sep 2005 to Apr 2008.
        \begin{itemize}
            \item[] {\bf Mentor} Prof. {Guo Li}
        \end{itemize}\\
        & {\bf Beijing University of Posts and Telecommunications, Bachelor of engineering} \\
        & School of Information Engineering, Information Engineering, From Sep 2001 to Jun 2005. \\\\
\end{tabularx}

\begin{tabularx}{\textwidth}{lX}
    \makebox[.12\textwidth][l]{\large \bf Interests} & photography, travel, trekking, cycling \\
\end{tabularx}

\end{document}

% vim:set spell:
