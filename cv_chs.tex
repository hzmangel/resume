\documentclass[12pt, a4paper]{article}

\usepackage[boldfont, slantfont, CJKchecksingle]{xeCJK}
\usepackage{palatino}
\usepackage{tabularx}
\usepackage{booktabs}
\usepackage{multirow}
\usepackage{fancyhdr}
\usepackage{xltxtra}
\usepackage{enumitem}
\usepackage[margin=.8in]{geometry}
\linespread{1.05}

\defaultfontfeatures{Mapping=tex-text}
\setCJKfamilyfont{yahei}{Microsoft YaHei}
\setCJKfamilyfont{heiti}{Adobe Heiti Std}
\setCJKfamilyfont{kaiti}{Adobe Kaiti Std}
\setCJKfamilyfont{georgia}{Georgia}

\newcommand{\yahei}{\CJKfamily{yahei}}
\newcommand{\heiti}{\CJKfamily{heiti}}
\newcommand{\kaiti}{\CJKfamily{kaiti}}
\newcommand{\georgia}{\CJKfamily{georgia}}

\pagestyle{fancy}
\fancyhead{}
\fancyfoot{}
\fancyfoot[C]{\scriptsize\georgia Resume Generated by \XeLaTeX}

\setCJKmainfont[BoldFont=Adobe Heiti Std]{Adobe Song Std}
%\setmainfont{Georgia}
\setmainfont{Calibri}

\begin{document}

\thispagestyle{empty}

\begin{center}
\begin{tabular}{lr}
    \multirow{3}{*}{\makebox[.05\textwidth][l]{}\makebox[.55\textwidth][l]{\Huge \kaiti 胡子明}} & %
        \makebox[.35\textwidth][l]{电话:(+86)1861-832-8360} \\
      & \makebox[.35\textwidth][l]{邮箱:hzmangel@gmail.com} \\
      %& \makebox[.35\textwidth][l]{主页:http://blog.hzmangel.info} \\
  \bottomrule
\end{tabular}
\end{center}

\centering

\begin{tabularx}{\textwidth}{lX}
    \makebox[.12\textwidth][l]{\large \kaiti 个人简介} &%
    4年高性能计算调度软件LoadLeveler的开发经验,熟悉Linux系统编程。%
    3年Web相关开发经验。在Web开发中偏向后端开发,也可完成简单的前端页面的编写和调试。%
    可独立完成基于Ruby On Rails站点前后端程序的编写调试。%
    熟练使用Python,并将其应用于日常的工作脚本中。%
    熟练使用Golang,曾独立设计并实现站点爬虫系统。%
    熟练使用MongoDB,Redis,有实际站点性能优化的经验。%
    较熟悉Android,iOS开发,\\\\
\end{tabularx}

\begin{tabularx}{\textwidth}{lX}
    \makebox[.12\textwidth][l]{\large \kaiti 专利} &%
    在IBM工作期间发表专利 {\it Scheduling Jobs In a Cluster}, 专利号 {\bf US 2010/0223618 A1}\\\\
\end{tabularx}

\begin{tabularx}{\textwidth}{lX}
    \makebox[.12\textwidth][l]{\large \kaiti 当前工作} & {\bf Favorite Medium 高级软件工程师} (2012年4月至今) \\
\end{tabularx}

\setlength{\leftskip}{.16\textwidth}
{\bf IBM高级软件工程师} (2010年4月至2012年4月)\\
从事IBM高性能计算调度器软件LoadLeveler相关的开发与维护工作, 较多的参与新功能设计讨论,参与制定新功能的开发进度。%
同时为客户提供L3技术支持。

{\bf IBM软件工程师} (2008年4月至2010年4月)\\
从事IBM高性能计算调度器软件LoadLeveler相关的开发与维护工作, 主要负责系统配置方式改进及底层网络通信相关方面的开发。
为产品增加了共享内存,数据库和IPv6的支持,并设计实现了一套新的用于问题分析的系统并成功与其它产品结合。
此外,还在L2团队的帮助下,还积极帮助客户解决所面临的问题。

\begin{description}[leftmargin=\leftskip, labelindent=.12\textwidth]
    \item[2012.8至2013.4] Appcara项目,

    \item[2012.1至2012.4] 设计并实现新特性用以更改运行状态下任务所占资源的分配以满足客户的需求。
        产品的现有功能只能更改处于非运行状态的任务,而此改进将会允许改变运行中任务的状态。
        在此项目中,我负责在任务管理进程端完成更改参数的功能,
        并保证后续预调度任务参量的重新计算。项目目前正在进行中。

    \item[2011.6 -- 2012.1] 参与设计并使用lxc重新实现了产品在Linux中对任务的断点保存及恢复,
        并在此基础上完成了任务的动态迁移。新功能移除了产品中使用第三方库实现类似功能的代码,
        在降低了产品依赖性的同时也提高产品的稳定性。
        我在中后期加入项目,主要负责设计实现迁移过程中节点之间任务及数据的传输,
        并在完成此部分功能后,辅助实现了任务调度进程中相关功能的代码。

    \item[2010.4 -- 2011.4] 设计并实现了一套不同于现有日志系统的机制用于查看软件运行状态。
        此机制在现有的日志系统之外新增了一些存放于单独文件或数据库中的消息,
        并提供一个Python的脚本用于分析查询以及格式化输出此类特殊的消息。
        此次改进为系统提供了一种更加便捷的跟踪定位问题的方法,可以有效的增加软件的易用性。
        并且此功能将本产品嵌入到蓝色基因(Blue Gene)集群的问题分析框架内,可以更有效的为客户提供服务。

    \item[2009.10 -- 2010.3] 通过对底层支撑库的改进,设计并实现了系统对IPv6协议的支持。
        通过此次改动,系统可以成功运行在纯IPv6环境以及IPv4/v6双栈环境中,拓宽了系统的使用范围,
        去除了由于IP地址不足导致集群规模受限的问题。此项目由我独立完成设计及编码,所以在项目进行的过程中,
        曾尝试引入敏捷编程的部分实践,如测试驱动开发,随时重构,动态联编等,并取得了较明显的效果。
        在项目完成后在组内组织过经验分享会。

    \item[2008.12 -- 2009.9] 负责设计并实现使用数据库存储配置信息并通过共享内存加速配置信息获取的子模块。
        此外还参与了数据表迁移,数据库内容写入等功能的实现。相比于原有系统使用的文件模式,
        此方法能减少集群部署的复杂性,并提供基于网页形式的配置信息查询更新工具。
        除此以外还为大规模集群中可能出现的突发读取实现了一种新的配置分发模式,为配置服务器去中心化提供了一种解决方案,
        一方面可以减少配置服务器的负载,另一方面也可以增强集群对单点错误的容错能力。\\
\end{description}


\setlength{\leftskip}{0pt}
\begin{tabularx}{\textwidth}{lX}
    \makebox[.12\textwidth][l]{\large \kaiti 其它工作} & 在业余时间,%
    我还会根据兴趣参与一些活动以提高自身的知识水平。主要参加过的活动有: \\
\end{tabularx}

\setlength{\leftskip}{.16\textwidth}
\begin{description}[leftmargin=\leftskip, labelindent=.12\textwidth]
    \item[社区活动] 北邮linuxfb社区月度活动的发起者与组织者,整理并发布每月聚会话题的幻灯片等工作。
        此外还使用Django为社区制作了一个信息发布网站 (http://linuxfb.net)。

    \item[自学及分享] 常年在业余时间学习新技术,并经常在社区月度活动上分享经验或在blog上发表相关文章。
        分享的话题包含有LaTeX, Python, GAE, Hadoop, Django, Haskell等。

    \item[Linux系统管理] 从2009年至今,担任组内系统管理员。负责系统配置管理维护,虚拟化平台搭建等工作。

    \item[翻译]参与了社区组织的翻译 The Django Book 的活动。

    \item[OCFS2]为OCFS2项目提交了两个补丁,用于修复Python方面的问题(BNC\#476388 和 BNC\#448523)

    \item[YAST2]为yast2-multipath软件包编写了Python的代理模块。

    \item[fcache]在校期间,和同伴完成移植加快Linux系统启动速度的fcache补丁到ext4文件系统,
        并完成性能测试工作 (参见: http://goo.gl/eAYev)。补丁被应用于第二年Google Summer of Code的一个项目中。
\end{description}

\setlength{\leftskip}{0pt}
\begin{tabularx}{\textwidth}{lX}
    \makebox[.12\textwidth][l]{\large \kaiti 教育背景} & {\bf 北京邮电大学,硕士学位} \\
        & 信息工程学院工学硕士,信号与信息处理专业,2005年9月至2008年4月
        \begin{itemize}
            \item[$\bullet$] 导师:{\kaiti 郭莉} 教授
        \end{itemize}\\
        & {\bf 北京邮电大学,学士学位} \\
        & 信息工程学院工学学士,信息工程专业,2001年9月至2005年6月 \\\\
\end{tabularx}

\begin{tabularx}{\textwidth}{lX}
    \makebox[.12\textwidth][l]{\large \kaiti 兴趣爱好} & 摄影,旅行,骑行 \\
\end{tabularx}

\end{document}
