\documentclass{resume} % Use the custom resume.cls style

\usepackage[boldfont, slantfont, CJKchecksingle]{xeCJK}
\usepackage{palatino}
\usepackage{tabularx}
\usepackage{booktabs}
\usepackage{multirow}
\usepackage{fancyhdr}
\usepackage{xltxtra}
\usepackage{enumitem}
\usepackage[margin={.6in}]{geometry}
\linespread{1.05}

\usepackage[xetex,plainpages=false,pdfstartview=FitH]{hyperref}

\defaultfontfeatures{Mapping=tex-text}
\setCJKfamilyfont{yahei}{Microsoft YaHei}
\setCJKfamilyfont{heiti}{Adobe Heiti Std}
\setCJKfamilyfont{kaiti}{Adobe Kaiti Std}
\setCJKfamilyfont{georgia}{Georgia}

\makeatletter
\def\myrev{Next Review Date:}
\def\ps@lastpage{%
  \let\@oddhead\@empty\let\@evenhead\@empty%
  \def\@oddfoot{{\slshape\myrev}\hfil\thepage}%
  \def\@evenfoot{{\slshape\myrev}\hfil\thepage}%
  }
\makeatother

\pagestyle{empty}
\fancyhead{}
\fancyfoot[C]{\small Resume Generated by \XeLaTeX}

\newcommand{\http}{http:/\hspace{-0.3ex}/}

\newcommand{\yahei}{\CJKfamily{yahei}}
\newcommand{\heiti}{\CJKfamily{heiti}}
\newcommand{\kaiti}{\CJKfamily{kaiti}}
\newcommand{\georgia}{\CJKfamily{georgia}}

\setCJKmainfont[BoldFont=Adobe Heiti Std]{Adobe Song Std}
%\setmainfont{Georgia}
\setmainfont{Calibri}
\setmonofont{Inconsolata}

\begin{document}

\thispagestyle{empty}

\begin{tabular}{lr}
    \multirow{3}{*}{\makebox[.05\textwidth][l]{}\makebox[.55\textwidth][l]{\Huge \kaiti 胡子明}} & %
        \makebox[.35\textwidth][l]{电话:(+86)1861-832-8360} \\
      & \makebox[.35\textwidth][l]{邮箱:hzmangel@gmail.com} \\
      & \makebox[.35\textwidth][l]{{博客}:\href{http://hzmangel.github.io/}{\tt \http{}hzmangel.github.io/ }} \\
\end{tabular}

%%%%%%%%%%%%%%%%%%%%%%%%%%%%%%%%%%%%%%%%%%%%%%%%%%%%%%%%%%%%%
% 个人简介
%%%%%%%%%%%%%%%%%%%%%%%%%%%%%%%%%%%%%%%%%%%%%%%%%%%%%%%%%%%%%
\begin{rSection}{\kaiti 个人简介}

高级软件工程师,供职于FavoriteMedium,后升职为Principal Software Engineer。主要专注于服务器后端服务的技术选型,架构以及开发,在项目中担任过项目主管,帮助管理项目进度以及和客户沟通。有前端经验,可以使用jQuery和AngularJS等技术完成简单的前端页面。

此前就职于IBM,负责开发高性能计算调度软件LoadLeveler,具有4年C++开发经验,主要负责计算任务调度分配和Linux网络编程相关领域。

此外,有多于10年的Linux使用及维护经验,熟练使用*nix工具如sed。在开发中可以根据需要选取C/C++,Ruby,Python,Golang,MongoDB和Redis等技术,有使用MongoDB,Redis和Docker创建高性能Web服务的经验。对于技术始终怀有兴趣及热情,并有有较强的学习能力。

\end{rSection}

%%%%%%%%%%%%%%%%%%%%%%%%%%%%%%%%%%%%%%%%%%%%%%%%%%%%%%%%%%%%%
% 专利
%%%%%%%%%%%%%%%%%%%%%%%%%%%%%%%%%%%%%%%%%%%%%%%%%%%%%%%%%%%%%
\begin{rSection}{\kaiti 专利}
在IBM工作期间发表专利 {\it Scheduling Jobs In a Cluster}, 专利号 {\bf US 2010/0223618 A1}
\end{rSection}

%%%%%%%%%%%%%%%%%%%%%%%%%%%%%%%%%%%%%%%%%%%%%%%%%%%%%%%%%%%%%
% Education
%%%%%%%%%%%%%%%%%%%%%%%%%%%%%%%%%%%%%%%%%%%%%%%%%%%%%%%%%%%%%
\begin{rSubsection}{在线课程}{}{}{}
  \begin{itemize}
    \item 海量数据挖掘(Cousera),\textbf{优秀}。
  \end{itemize}
\end{rSubsection}

\begin{rSection}{\kaiti 教育背景}
\begin{rSubsection}{北京邮电大学}{2005年9月 - 2008年4月}{信号与信息处理,硕士学位}{}
\end{rSubsection}

\begin{rSubsection}{北京邮电大学}{2001年9月 - 2005年6月}{信息工程,学士学位}{}
\end{rSubsection}
\end{rSection}

%%%%%%%%%%%%%%%%%%%%%%%%%%%%%%%%%%%%%%%%%%%%%%%%%%%%%%%%%%%%%
% 当前工作
%%%%%%%%%%%%%%%%%%%%%%%%%%%%%%%%%%%%%%%%%%%%%%%%%%%%%%%%%%%%%
\begin{rSection}{\kaiti 当前工作}

\begin{rSubsection}{Favorite Medium}{2012年4月 - 至今}{Principal Software Engineer}{}{}
\begin{rSubsectionList}
\item 以高级软件工程师加入公司,后升职为Principal Software Engineer。
\item 与客户沟通,明确客户的需求和问题。
\item 选择合适的语言和技术(Ruby, Python, Golang, Docker等)构建符合需求的系统。
\item 有较丰富的后台服务开发经验,有多种语言/框架,如Ruby On Rails, koajs, Flask, Golang等,的实际项目经验。
\item 在某些项目中负责全栈开发。
\item 有网络爬虫,数据挖掘和分析的相关经验。
\item 研究新技术并在组内做分享。
\end{rSubsectionList}
\end{rSubsection}
\end{rSection}

%%%%%%%%%%%%%%%%%%%%%%%%%%%%%%%%%%%%%%%%%%%%%%%%%%%%%%%%%%%%%
% 过往项目和经验
%%%%%%%%%%%%%%%%%%%%%%%%%%%%%%%%%%%%%%%%%%%%%%%%%%%%%%%%%%%%%
\begin{rSection}{\kaiti 过往项目和经验}

\begin{rSubsection}{Starcount}{2013年2月 - 至今}{架构及开发}{}

Starcount (\href{http://www.starcount.com/}{\tt \http{}www.starcount.com/}) 是一个专注于计算公众人物流行度,并以此为公众人物打分,排序,预测的网站。它通过抓取公众人物在多个社交网站上的信息作为原始计算数据,通过MapReduce计算用户的评分。此项目包括主站,移动应用,后台CMS和数据爬虫。

\begin{rSubsectionList}
\item 设计并主导开发了后台系统,包括CMS,数据爬虫以及数据后期处理。
\item 实现并维护部分前端页面脚本代码。
\item 设计CMS架构并使用 \textbf{Ruby On Rails} 实现,帮助完成客户日常数据的处理,验证等操作。
\item 将之前使用的第三方数据计算平台替换为自建的 \textbf{ElasticSearch} + \textbf{Python} 结构,并更新相应的CMS和爬虫,为客户节省了开销。
\item 设计 \textbf{MongoDB} 数据存储结构,并使用改进数据结构及Index加速数据查询。提高查询速度 \textbf{50\%} 以上。
\item 重构爬虫程序。使用 \textbf{Golang} 重构之前基于 \textbf{Ruby} 的爬虫,使用非阻塞并发代替了循环,在抓取效率不变的情况下,将数据抓取节点数量从\textbf{30}台降低为\textbf{6}台。
\item 引入 \textbf{Redis} 及异步任务处理数据统计等长耗时任务,提升CMS端用户体验。
\end{rSubsectionList}
\end{rSubsection}


\begin{rSubsection}{Steelcase Scout}{2015年4月 - 至今}{开发及系统管理}{}

Steelcase Scout是一个使用智能传感器网络来解决复杂开放办公环境下空间及资源预订问题的服务。

\begin{rSubsectionList}
\item 组织并维护提供服务的 \textbf{docker} 容器,极大方便了不同环境中服务的部署和迁移。
\item \textbf{Tornado} API服务的主要开发人员,设计并实现了预约,空间检查及管理API接口。并在开发任务之外协助PM处理沟通事宜。
\item 为移动App,传感器数据设计DB以及Big Query架构。完成客户的查询及数据统计需求。
\item 架设 \textbf{Jenkins} 服务器帮助持续集成,方便Docker容器的更新。
\item 使用 \textbf{Gatling} 进行压力测试,并根据测试结果调整代码以提高性能。
\end{rSubsectionList}
\end{rSubsection}


\begin{rSubsection}{LoadLeveler}{2008年4月 - 2012年4月}{高级软件工程师 (\textit{从2011年6月起})}{}

参与设计和开发高性能计算调度系统LoadLeveler。LoadLeveler是使用C++编写的,运行于Linux/AIX上的调度软件,它可以在大规模的集群,如蓝色基因(Blue Gene)中,有效的调度计算任务,最大化系统资源的利用。我主要负责了以下几方面的开发:

\begin{rSubsectionList}
\item 增强配置文件:改进配置文件系统,将配置从文件中迁移到数据库中,以提高在大型集群中的系统效率。
\item IPv6支持:使用新的系统调用并更新系统内部数据结构,使得系统可以运行于纯IPv6,纯IPv4以及IPv6/v4混合环境中。
\item 任务迁移:使用cgroup的相关系统调用,实现任务在Linux平台上的动态迁移。
\item 动态改变任务占用资源:支持在任务运行中更改所占用资源,增加系统资源使用率。
\end{rSubsectionList}\vspace{-1.5em}

此外,我还负责开发组内测试集群的部署和维护。
\end{rSubsection}
\vspace{1.8em}


\begin{rSubsection}{3D打印的切片系统}{2015年3月 - 2015年5月}{开发及运维}{业余项目}

此系统用于3D打印中切片任务的管理。用户调用接口后,系统启动docker容器进行切片,并实时更新状态信息到数据库中。此外还提供了查询接口用以查询任务运行状态。

\begin{rSubsectionList}
\item 使用 \textbf{Golang} 设计并实现API服务器。
\item 设计多个容器间消息传递的格式。
\item 使用 \textbf{Docker} 构建系统,并使用Golang API管理容器。
\item 使用 \textbf{Redis PUB/SUB} 模型在容器间异步传递消息。
\end{rSubsectionList}
\end{rSubsection}



\begin{rSubsection}{五味厨房电商网站}{2013年3月 - 2013年9月}{全栈开发工程师}{业余项目}

五味厨房是一个专注于生鲜配送的电商网站。网站提供了电商网站所需要的基本功能,如用户管理 ,购物车,订单管理,优惠码,充值券,用户余额管理等,并提供了一套完善的后台以管理商品,订单,物流,库存,优惠等相关内容。网站的第一个版本是基于Ruby On Rails开发的,后续升级改用了AngularJS以及ExpressJS.

\begin{rSubsectionList}
\item 设计并实现了 \textbf{MongoDB} 数据库的表结构和查询。
\item 设计并实现了电子商务服务接口。
\item 设计并实现了前端页面及管理后台。
\item 在系统中集成支付宝的基本支付接口。
\end{rSubsectionList}
\end{rSubsection}

\end{rSection}

\end{document}
